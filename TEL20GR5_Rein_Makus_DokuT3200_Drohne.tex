\documentclass[ a4paper,
                oneside,
                toc=bibliography,
                toc=listof
                ]{scrbook}

% \usepackage[ngerman, english]{babel} % If the thesis is in English
\usepackage[english, ngerman]{babel} % If the thesis is in German


%Farbdefinitionen
\usepackage[hyperref,dvipsnames]{xcolor}

%%%
% Required for custom acronyms/glossaries style
% Left aligned Columns in tables with fixed width
% see http://tex.stackexchange.com/questions/91566/syntax-similar-to-centering-for-right-and-left
\usepackage{ragged2e}
%%%

%%%
% Abkürzungsverzeichnis
\usepackage{scrwfile} % Wichtig, ansonsten erscheint "No room for a new \write"

% von https://tex.stackexchange.com/questions/73160/table-with-tabularx-and-multirow
\usepackage{graphicx}
\usepackage{subfig}
\usepackage{tabularx}
\usepackage{multicol}
\usepackage{float}
%\setlength\columnseprule{0.5pt}

%%%
% ggf.in der Endversion komplett rausnehmen. dann auch \href in commands.tex aktivieren
% Alle Optionen nach \hypersetup verschoben, sonst crash
%
%\usepackage[]{hyperref}%siehe auch: "Praktisches LaTeX" - www.itp.uni-hannover.de/~kreutzm
%
%% Da es mit KOMA 3 und xcolor zu Problemen mit den global Options kommt MÜSSEN die Optionen so gesetzt werden.
%
%\ifdeutsch
\usepackage[ngerman]{hyperref}

% siehe http://www.dickimaw-books.com/cgi-bin/faq.cgi?action=view&categorylabel=glossaries#glsnewwriteexceeded
\usepackage[acronym,indexonlyfirst,nomain]{glossaries}
\renewcommand*{\acronymname}{Abkürzungsverzeichnis}
\renewcommand*{\glsgroupskip}{}
\makenoidxglossaries

\usepackage[ngerman,capitalise,nameinlink,noabbrev]{cleveref}
\crefname{section}{Kapitel}{Kapitel}
%\else
%\usepackage[capitalise,nameinlink,noabbrev]{cleveref}
%\fi

\usepackage[chapter]{minted}
\newmintedfile[pycode]{python}{linenos,breaklines,
               numbersep=5pt,
               frame=lines,
               framesep=2mm}
\newmintedfile[cccode]{cpp}{linenos,breaklines,
               numbersep=5pt,
               frame=lines,
               framesep=2mm}

% Eigene Farbdefinitionen ohne die Namen des xcolor packages
\definecolor{darkblue}{rgb}{0,0,.5}
\definecolor{black}{rgb}{0,0,0}

\hypersetup{
    breaklinks=true,
    bookmarksnumbered=true,
    bookmarksopen=true,
    bookmarksopenlevel=1,
    breaklinks=true,
    colorlinks=true,
    pdfstartview=Fit,
    pdfpagelayout=TwoPageRight, % zweiseitige Darstellung: ungerade Seiten rechts im PDF-Viewer - siehe auch http://tex.stackexchange.com/a/21109/9075
    filecolor=darkblue,
    urlcolor=darkblue,
    linkcolor=black,
    citecolor=black
}
%
%%%

% erweitertes Enumerate, NICHT paralist
\usepackage{enumitem}
\setlist[enumerate]{noitemsep}
\setlist[itemize]{noitemsep}
\setlist[description]{noitemsep}


% This class does the ISW styling for you (together with scrbook).
%
% It handles the following:
% - Proper input and font encoding (Just type, don't care about the LaTeX compiler you use or how to type German umlauts)
% - Fonts with ligatures and kerning (Tex Gyre fonts are used, part of every LaTeX installation, text is nice to read)
% - Bibliography styling for biblatex (declare your bibliography file and you are ready to go)
% - Provide command for title page (\makeISWtitle) and declaration of originality ( \declarationOfOriginality)
% - Loads packages "biblatex" and "graphics"
\usepackage[
    type=studie, %master, % bachelor, study, bachelorproject
]{iswthesis}

%Path to .bib-File(s) for BibLatex
\addbibresource{bibliography.bib}
\addbibresource{Studienarbeit_T3200.bib}
% \addbibresource{someOtherBibFile}
% Hier stehen alle Abkürzungen
\newacronym{er}{ER}{error rate}
\newacronym{fr}{FR}{Fehlerrate}
\newacronym[plural={RDBMS},shortplural={RDBMS}]{rdbms}{RDBMS}{Relational Database Management System}


\newacronym{uas}{UAS}{Unmanned Aircraft System}
\newacronym{uav}{UAV}{Unmanned Aerial Vehicle}
\newacronym{gps}{GPS}{Global Positioning System}
\newacronym{imu}{IMU}{inertiale Messeinheit}
\newacronym{rpi}{RPI}{Raspberry Pi (Einplatinencomputer)}
\newacronym{tcp}{TCP}{Transmission Control Protocol}
\newacronym{udp}{UDP}{User Datagram Protocol}
\newacronym{gcs}{GCS}{Ground Control Station}
\newacronym{ram}{RAM}{Arbeitsspeicher}
\newacronym{lba}{LBA}{Luftfahrt-Bundesamt}
\newacronym{mav}{MAVLink}{\textit{MAVLink}}
\newacronym{hil}{HIL}{Hardware in the Loop}
\newacronym{gui}{GUI}{Graphical User Interface}

\newglossaryentry{slam}
{
    name=SLAM,
    short={SLAM},
    description={Simultane Lokalisierung und Kartierung der Umgebung}
}
\newglossaryentry{px4}
{
    name=Pixhawk® 4,
    short=PX4,
    description={Microcontroller, mit Software und Peripherie zur Flugsteuerung}
}
\newglossaryentry{ros}
{
    name=ROS,
    short=ROS,
    description={Robot Operating System, Framework mit Schnittstellen für Sensor/Aktor-Verknüpfungen}
}

\author{Markus Rein}
\placeOfBirth{Leipzig}
\address{Alexanderstraße 146, 70180 Stuttgart}
\major{Elektrotechnik}
\title{Erweiterung bestehender Drohnen um eine Autonomflugfähigkeit II}
% \titleTranslated{Wie man einen Hamster trainiert}
\matrnr{6983030, TEL20GR5}
\date{23.01.2023 -- 14.05.2023}
\supervisor{Prof. Dr.-Ing. Johannes Moosheimer}
% \professor{Prof. Dr.-Ing. Oliver Riedel}
\usepackage{atbegshi}% http://ctan.org/pkg/atbegshi
\AtBeginDocument{\AtBeginShipoutNext{\AtBeginShipoutDiscard}} 
    
\begin{document}
    \frontmatter
    \makeISWtitle
    
    \cleardoublepage
	\setcounter{page}{1} % start at page (i) after title page
    \declarationOfOriginality

    % Kurzfassung/Abstract
\chapter*{Abstrakt}
In dieser Arbeit wird eine Modellbaudrohne mit Sensoren erweitert um einen autonomen Flug durchzuführen.

Tiefgehende Versuche werden durchgeführt um die Fähigkeiten der Drohne zu beurteilen. Das ursprüngliche System verfügt bereits über ausgiebige Funktionen die einem autonomen Flug sehr nahe kommen.

Die Drohne wird mit einem zusätzlichen Computer ausgestattet, der es erlaubt selbige per Netzwerkzugriff zu steuern. Gleichzeitig kann dieser zusätzliche Sensoren auslesen und Auswerten. Ultraschallsensoren und Kameras sollen zum Erkennen von Hindernissen zum Einsatz kommen. Eine Software zur Planung von Flugbahnen mit Wegkorrektur beim Erscheinen von Hindernissen wird getestet und auf das Projekt angepasst. Die Auswertung der Software erfolgt auf der Drohne selbst, sodass diese ohne weiteres Zutun ihren Weg anpassen kann.

Im Projekt wird das Einlesen der Ultraschallsensoren ausgiebig entwickelt. Die Verwendung von Kameras ist vorgesehen um eine Tiefenkarte der Umgebung zu erzeugen, scheitert jedoch an der Anbindung an eine vorgesehene Hardware. Jedoch kann die Drohne allein mit den Ultraschallsensoren um einfache Hindernisse herum navigieren. 

\chapter*{Abstract}
In this thesis a model construction drone is equipped with additional sensors to perform an autonomous flight.

Far-reaching tests were conducted to assess the capabilities of the drone. The original system already implements extensive functionality which comes close to autonomous flight.

During the project the drone is extended by an additional computer which allows to control the drone per network interface. At the same time this computer can interface additional sensors and process data. Ultrasonic sensors and cameras shall be used for detection of obstacles. A software to adopt a trajectory based on occuring obstacles is tested and tailored to the project. All processing is performed on-board the drone so that it can fly without manuel interventions.

The project documents development for interfacing ultrasonic sensors. The usage of cameras is intended to generate a depth-map of the surroundings, though failed due to connectivity of the cameras. But still the drone is able to fly solely using ultrasonic sensors to navigate around small obstacles.

    \cleardoublepage
    \tableofcontents

    \mainmatter

    % ********************************************************************
    % Write your own contents here:
    % ********************************************************************
    \chapter{Einleitung}
Dieses Projekt dient der Verwirklichung Autonomen Fliegens.

\textbf{Ziele der Arbeit:}
\begin{itemize}
    \item Erweiterung der Sensorfähigkeit der Drohne um Tiefenkamera
    \item Verwendung von Tiefenbildern und Ultraschallsensordaten für Hinderniserkennung 
    \item Autonomer Flug der Drohne um große, flächige Hindernisse wie Häuser oder Automobile unter Verwendung der Software \textit{Avoidance}
\end{itemize}

\textbf{Optionale Erweiterungen:}
\begin{itemize}
    \item Verfeinerungen und Anpassungen der Software \textit{Avoidance}, sodass kleinere, strukturreiche Hindernisse (bspw. Bäume) erkannt und umflogen werden
    \item Einführung kamerabasierte Zielerkennung als Alternative zur GPS-gestützten Zielführung
\end{itemize}

\textbf{Geplantes Vorgehen:}
\begin{enumerate}
    \item Vervollständigen der Simulation aus vorhergehendem Projektteil: die simulierte Drohne soll im Flug allein mit Kamerabildern und der Software \textit{Avoidance} Hindernissen ausweichen 
    \item Beschaffung und Installation Tiefenkamera an realer Drohne
    \item Aufspielen der Software \textit{Avoidance} auf reale Drohne
    \item Flugtests mit Tiefenkamera
    \item Einbinden von Ultraschallsensordaten in Softwareverarbeitung von \textit{Avoidance}
    \item Flugtests mit Tiefenkamera und Ultraschallsensordaten
    \item mehr Details: was funktioniert in der Endanwendung, was sind die Schritte in der Simulation, ,was wird simuliert
\end{enumerate}

Im weiteren Verlauf der Arbeit spielen folgende Begriffe eine wichtige Rolle:\\
\begin{minipage}[t]{\linewidth}
\begin{description}
    \item[Bodenstation:] 
    \item[\gls{imu}:] Zusammenfassung der Sensorn Beschleunigungssensor, Kompass und Gyroskop
    \item[\gls{rpi}:] Einplatinencomputer, eingesetzt als Bordcomputer
    \item[\gls{ros}:] Metabetriebssystem zur Vernetzung von Roboterbestandteilen (Aktor-Sensor-Verknüpfung), Überwachung und Steuerung von Prozessen
    \item[Docker:] Containerisierung von Betriebssystemumgebungen. Beispielsituation:
    \begin{itemize}
        \item auf dem \gls{rpi} läuft Raspberry Pi OS (Linux Debian), genannt Host-Betriebssystem
        \item unabhängig davon kann innerhalb eines Containers Linux Ubuntu installiert werden
        \item im Container sind speziell festgelegte Bibliotheken, Umgebungsvariablen, Speichergeräte und Netzwerke zum Programmbetrieb vorhanden, es können Programme gestartet werden, die nicht auf dem Host-Betriebssystem lauffähig sind
        \item im Container gestartete Programme laufen parallel zu Programmen auf dem Host-Betriebssystem, sie sind trotzdem im Prozessmanager (\textit{top}) von Raspberry Pi OS sichtbar
    \end{itemize}  
    \item[PX4:] Offizielle Software der Dronecode Stiftung mit der verschiedene $\mu C$, embedded System, oder PC zum Steuern von Robotern, Fahrzeugen oder Drohnen ausgestattet werden können
    \item[Avoidance:] Projekt im Umfeld von PX4 zur autonomen Steuerung von Drohnen in unbekanntem Umfeld. Aufgeteilt in 3 Unterprojekte:
    \begin{itemize}
        \item Local Planner
        \item Global Planner
        \item Safe Landing Planner  
    \end{itemize}
    Im derzeitigen Projekt wird der Local Planner umgesetzt.
    \item[...]
\end{description}
\end{minipage}
    \chapter{Einführung in Projekt}
Auf physikalische Zusammenhänge wird in der Arbeit nicht eingegangen, da am bestehenden System Drohne keine Änderungen vorgenommen werden. Es wird lediglich die Software angepasst, was die Flugfähigkeit aber nicht beeinflußt.

\section{Verwendete Technologien}
\subsection{WLAN}
\subsection{GPS}

\section{Beschreibung der Drohne in den einzelnen Phasen}
\subsection{Beschreibung des Ausgangsystems Drohne}
Die Drohne kann mit Methoden der Regelungstechnik als adaptives System beschrieben werden. Dabei dient der Flugcontroller als Regler, die Motoren als Steuerstrecke, und Bordcomputer sowie Bodenstation zur Identifikation und Modifikation der Parameter. Zu Beginn wird die Drohne in ihrer Ausgangslage, ohne automone Flugfähigkeiten beschrieben. Die Bestandteile des Systems sind zunächst in Tabelle \ref{tab:system_intro} aufgelistet.

\begin{table}[!ht]
    \caption{Systemübersicht Drohne und Bodenstation}
    \begin{tabularx}{\textwidth}{l | X | X | X }
    & \multicolumn{2}{c |}{Drohne} & \\
    & Flugcontroller & Bordcomputer & Bodenstation \\ \hline
    Funktion & Autopilot-Software liest Sensoren und steuert Motoren der Drohne & Bereitstellung \acrshort{wlan}-Netzwerk zur Verbindung von Autopilot und Bodenstation & Parametrierung und Steuerung der Drohne\hfill \\ \hline
    Hardware & Pixhawk 4 & Raspberry Pi 3B+ & PC und/oder Smartphone \hfill \\ \hline
    Software & PX4 & MAVLink-Router% \newline -ROS-Umgebung \newline -Avoidance \newline Hindernisse, Trajektorie, Flugcontroller bedienen, Ultraschallsensoren
    & QGroundControl \hfill \\
    \label{tab:system_intro}
    \end{tabularx}
\end{table}

Zusammen bilden sie ein System wie in Bild \ref{fig:system_intro} gezeigt. Im Flugbetrieb werden von der Bodenstation Flugbefehle (feste Zielkoordinaten, relative Koordinaten, oder manuelle Motoransteuerung) per \acrshort{wlan}-Verbindung an den Bordcomputer, von diesem per Serieller Schnittstelle (\acrshort{uart}) an den Flugcontroller, geschickt. Mehr Details zur Steuerung mit der Bodenstation in \cref{chap:intro_capabilities}. Der Flugcontroller steuert anschließend die Motoren um die gewünschte Position zu erreichen. Als Eingabegrößen stehen dem Flugcontroller Sensordaten von Beschleunigungssensor, Kompass (erstere bilden zusammen die \gls{imu}) und Barometer, diese sind im Flugcontroller integriert, und der GPS-Antenne zur Verfügung.

\subsection{Beschreibung des Zielsystem Drohne mit autonomer Flugfähigkeit}
Zusätzliche Sensoren stellen Daten für Berechnungen auf dem Bordcomputer bereit. Diese müssen ausgewertet und die Ergebnisse dem Flugcontroller zugespielt werden. Das Auswerten und Zuspielen der Daten ist zeitkritisch denn es beeinflußt direkt den Flug der Drohne. Im Idealfall sollten Berechnungen direkt auf dem Flugcontroller oder in unmittelbarem Zusammenhang mit diesem durchgeführt werden. Zu den Aufgaben zählt:
\begin{itemize}
    \item Erfassen von Ultraschall-Daten zum Detektieren von Hindernissen
    \item Erfassen von Bildern
    \item Verarbeiten von Bildern zur Detektieren von Hindernissen
    \item Berechnung alternativer Flugbahn zur Umgehung von Hindernissen
\end{itemize}

Die Umsetzung der Flugplanung soll mittels der Software \textit{Avoidance} erfolgen. Diese arbeitet auf dem Metabetriebssystem \acrshort{ros}, welches wiederum auf \textit{Ubuntu} aufsetzt. (Im Anwendungsfall ROS Noetic auf Ubuntu 20.04.) Die Software kann vollständig auf dem \gls{rpi} betrieben werden. Somit ergeben sich für den \gls{rpi} weitere Aufgabenbereiche, wie nachfolgend dargestellt. Das erweiterte System ist dargestellt in Bild \ref{fig:system_added_sensors}.

\begin{itemize}
    \item Ubuntu 20.04 als Betriebssystem (nur indirekt möglich innerhalb eines Containers möglich)
    \item \acrshort{ros}-Noetic innerhalb des Docker-Containers
    \item Einlesen der Ultraschallsensordaten, erzeugen von Tiefenkarte und publizieren entsprechender Topic
    \item Einlesen der Kamerabilder, erzeugen von Tiefenkarte und publizieren entsprechender Topic
    \item Ausführung von Avoidance
\end{itemize}

\subsection{Beschreibung der Simulation}\label{chap:intro_simulation}
Die Entwicklung von Software für und im Zusammenhang mit dem Flugcontroller sieht vor, in einer Simulation getestet zu werden. Von der Dronecode-Stiftung wird als offizielle Umgebung dazu der Simulator \textit{Gazebo} empfohlen\cite{dronecodestiftungSimulationPX4User}. Sowohl die \textit{PX4}- als die \textit{Avoidance}-Software können vollständig in diesem betrieben werden. Zum Betrieb des Simulators wird wieder das Betriebssystem Linux Ubuntu benötigt. Deshalb werden für das Projekt einige Einschränkungen aufgenommen. Die weiteren Ausführungen hier beschreiben die Ausgangslage zur Simulation, in \cref{chap:intro_avoidance} wird \acrshort{ros} eingeführt und eingerichtet.

Zum Vorgehen wird der \gls{hil}-Aufbau mit dem Simulator \textit{AirSim} von Microsoft\cite{microsoftcorporationWelcomeAirSim2023}, wie in \cite[Kapitel 3.4.1]{markusreinErweiterungBestehenderDrohnen2023} beschrieben, verwendet. Bild \ref{fig:system_sim} zeigt, in Anlehnung an Bild \ref{fig:system_intro}, die durch die Simulation übernommenen Funktionen. Die Komponenten sind im Betrieb wie folgt verbunden:
\begin{description}
    \item[PC-Flugcontroller:] \acrshort{usb}-Kabel, wird von \textit{AirSim} zur direkten Kommunikation mit dem Flugcontroller verwendet
    \item[PC-Bordcomputer:] \acrshort{wlan}, erlaubt Verbindung Bodenstation mit Flugcontroller
\end{description}

Die Aufgaben der Komponenten während der Entwicklung sind:
\begin{description}
    \item[Simulation:] In der Simulation werden sowohl alle physikalischen Effekte berechnet als auch die virtuelle Umgebung der Drohne dargestellt. Die Sensordaten werden der Drohne direkt von der Simulation eingespeist. Eine resultierende Ansteuerung der Motoren wird in die Simulation übernommen. Somit kann sich die Drohne in der Simulation wie in realer Umgebung bewegen. Außerdem werden von der Drohne aufgenommene, simulierte Kamerabilder bereitgestellt.
    \item[Drohne:] Der Flugcontroller auf der Drohne wird im \gls{hil}-Modus betrieben. Alle Ein- und Ausgänge zum Controller werden durch virtuelle Schnittstellen der Simulation ersetzt. Die Kommunikation mit dem Bordcomputer bleibt dieselbe wie zuvor, sodass die Drohne per Bodenstation gesteuert werden kann.
    \item[Bordcomputer:] Wird weiterhin nur zur Kommunikation zwischen Bodenstation und Flugcontroller verwendet. Erweiterte Funktionen werden auf einem separaten Rechner entwickelt und getestet um anschließend auf den Bordcomputer überspielt zu werden.
\end{description}

%\paragraph*{}
%Erweiterte Funktionalität wird auf dem Entwicklungsrechner in Containern, in Verbindung mit der Simulation, erprobt. Derartige fertige Anwendungen können dann direkt auf dem Bordcomputer eingesetzt werden. Bestandteile der Software sind:
%\paragraph*{}
%\begin{description}
%    \item[\gls{ros}] 
%    \item[\textit{mavros}:] auf gleicher Ebene angesiedelt wie eine Bodenstation, erlaubt Protokollübersetzung zwischen \gls{mav}- und \acrshort{ros}-Nachrichten für \acrshort{ros}-internen Datenaustausch, empfängt Daten der Drohne und sendet neue Anweisungen zur Drohne
%    \item[Tiefenverarbeitung:] arbeitet direk mit Tiefenbildern aus dem Simulator um eine \enquote{Punktwolke der Umgebung} zu generieren
%    \item[\textit{Avoidance}:] setzt neue Zielpunkte für Drohne anhand von Sensordaten der Drohne und Punktwolke von Kamera
%    \item[\textit{AirSim-Wrapper}:] nicht für Endanwendung benötigt, kommuniziert direkt mit dem Simulationsprogramm und stellt Tiefenbild bereit
%\end{description}
%
%
    \chapter{Stand der Technik}
In diesem Kapitel werden die Grundlagen verwendeter Software erläutert.
\section{Erweiterte Flugmodi der Drohne}\label{chap:intro_capabilities}
Bei Verwendung der Drohne in Verbindung mit einer Bodenstation, wird die aktuelle Position auf einer Karte eingezeichnet. Von diesem Punkt aus kann der Drohne eine Wegvorgabe eingespielt werden, der \enquote{Mission-Mode}. Dabei enthält die Karte Informationen zur ungefähren Beschaffenheit der Umgebung, sodass die Drohne nicht Tiefer fliegen würde als der Boden der Karte. Gleichzeitig sind die standardmäßigen Sicherheitsmaßnahmen eingestellt, bspw. eine Mindesflughöhe von $4m$ einzuhalten. Neben der Wegvorgabe können dem Flugcontroller weiterhin verbotene Zonen mitgeteilt werden, die nicht durchflogen werden dürfen, genannt \textit{\enquote{Geo-Fence}}. Der Algorithmus sieht derartige Zonen als Hindernis an. Bei Kontakt mit ihnen wird ein Failsafe ausgelöst. \textit{PX4} kennt zwei derartige Modi:
\begin{description}
    \item[Failsafe GeoFence:] Ein Zylinder dessen Durchmesser von der Funkreichweite der Fernbedienung und maximaler Flughöhe beschränkt ist. Bei Durchbruch verfällt die Drohne standardmäßig in den \enquote{Return-Mode} und kehrt zu ihrer Ausgangsposition zurück.
    \item[GeoFence Plan:] Kreise oder Polygone auf Karte die nicht durchflogen oder verlassen werden dürfen (je nach Einstellung). Bei Bruch der Bedingung verfällt die Drohne in den \enquote{Hold-Mode} und bleibt schlicht stehen.
\end{description}

Es ist also bereits mit Bordmitteln möglich das Flugverhalten zu beeinflussen. Für das Vorgehen mit \textit{Avoidance} kommt der \enquote{Offboard-Mode} zum Einsatz. In diesem Modus werden dem Flugcontroller ständig neue Anweisungen, als nächster Wegpunkt, eingespeist.

Weiterhin können der Drohne im Missionsmodus sogennante \gls{roi} mitgeteilt werden. Ist eine Kamera an der Drohne vorhanden, wird diese gezielt auf die Positionen gerichtet. Ist keine Kamera explizit definiert richtet sich die Drohne mit dem Bug in Richtung der \gls{roi} aus. Da die Drohne sowohl vorwärts als auch seitwärts fliegen kann, hält sie durchgehend auf den Punkt zu.

\section{ROS und Avoidance}
Das Projekt \enquote{Obstacle Detection and Avoidance}\cite{dronecodestiftungObstacleDetectionAvoidance2023}, auf GitHub verfügbar als PX4-Avoidance\footnote{\label{note1}\url{https://github.com/PX4/PX4-Avoidance}}, hier nur \textit{Avoidance} genannt, entstand in enger Zusammenarbeit mit der Dronecode Stiftung an der ETH Zürich, dem Ursprungsort aller \textit{PX4}-Software. Es arbeitet innerhalb einer \acrshort{ros}-Umgebung.

Es stehen im Projekt 3 Algorithmen zur Verfügung, die unabhängig voneinander zu betrachten sind. Alle dienen der Anpassung der Flugbahn in unbekannter Umgebung:
\begin{description}
    \item[Local Planner:] Navigiert um Hindernisse in der direkten Umgebung
    \item[Global Planner:] Speichert nahezu vollständige Karte der Umgebung und erlaubt Navigation durch Labyrinth-artige Umgebung
    \item[Safe Landing Planner:]
\end{description}

Die Software von \textit{Avoidance} erhält die Daten des Flugcontrollers über das Zwischenprogramm \textit{mavros} (\acrshort{mav}-zu-\acrshort{ros}-Übersetzung, siehe \cite[Kapitel 5.2/5.4]{markusreinErweiterungBestehenderDrohnen2023}). Es sind die Soll-Trajektorie und Sensordaten vom Flugcontroller bekannt. Außerdem wird zur Navigation eine \textit{Punktwolke} (siehe \cref*{chap:intro_pointcloud}) der Umgebung eingespeist. Falls das Programm ein Hindernis in der Flugbahn erkennt, wird eine angepasste Trajektorie an den Flugcontroller ausgegeben.

Die Software kann nicht direkt auf dem Flugcontroller ausgeführt werden, da die Berechnungen sehr viele Ressourcen (Rechenkapazität, Speicher) benötigen. Weiterhin empfehlen die Entwickler, zuerst den Local Planner zu implementieren, da dieser am besten funktioniert. Offizielle Empfehlungen der Entwickler verwenden leisstungsstarke Hardware wie Nvidia Jetson (Hardware-Unterstützung für Bildverarbeitung) oder Intel RealSense (Kamera mit Tiefenerkennung).

Im Zusammenhang mit der Software sind letztere bereits erprobt. Aufgrund des hohen Preises können sie nicht in diesem Projekt verwendet werden, siehe \cite[Kapitel 4.3.8]{wirthErweiterungBestehendenDrohne2022}. Als Alternative können auch Stereokameras verwendet werden. Beispielcode zur Einbindung von Tiefenbildern ist unter Github (siehe \cref{note1}) vorhanden. 

%Stereokamera liefert genaue Karte der Umgebung, ähnlich einem Lidar.
%Andere Methoden arbeiten eventuell nicht mit Avoidance zusammen. Doch doch

\section{Hinderniserkennung und ROS Punktwolken}\label{chap:intro_pointcloud}
Als Verschiedene Prinzipien stehen zur Hinderniserkennung zur Verfügung. Als Eingabegröße für \textit{Avoidance} müssen die verarbeiteten Bilder im Punktwolkenformat als \acrshort{ros}-Topic vorliegen.
%Quellen nicht eindeutig
%Der Fokus dieses Projektes, das Erkennen und Ausweichen von Hindernissen wird \enquote{Obstacle Avoidance} genannt. Es ist nicht zu verwechseln mit \enquote{Obstacle Detection}, dem Erkennen und Klassifizieren von Bildinhalten.
Nachfolgend vorgestellt werden die grundlegenden Techniken der Bilderkennung. 
\subsection{SLAM Algorithmus}\label{chap:slam}
\Gls{slam} Techniken entstanden bereits in den 1980-1990 Jahren und werden bspw. bei Robotern eingesetzt, die in Hallen navigieren (für die kein \acrshort{gps} verfügbar ist). Zum Einsatz kommen Kamerasysteme in Verbindung mit Entfernungssensoren (Sonar, Radar, Lidar). Die Ergebnisse von \gls{slam} können nicht garantiert werden und sind nicht reproduzierbar, weshalb es in keinen kritischen Umgebungen (bspw. wenn Verletzungsrisiko besteht) eingesetzt werden kann.

Allgemein wird \gls{slam} durch einen modularen Prozess beschrieben:
\begin{description}
    \item[Lokalisierung:] per Motorfortschritt, \gls{imu}, Kamera, etc.%\newline Bei v\gls{slam} kommen folgende Prinzipien zum Einsatz:
    \item[Kartengenerierung:] durch einen der Algorithmen
\begin{itemize}
    \item Markov-Lokalisierung: Wahrscheinlichkeit des Aufenthaltsortes wird angenommen und über Zeit verfeinert; Iterativ; Ressourcenaufwendig
	\item Kalman-Filter: Ermöglicht basierend auf Sensordaten schnelles wiederfinden aktueller Position; anfällig bei Verlust von Eingangsdaten
	\item Monte-Carlo-Lokalisierung (Partikelfilter): nimmt Wahrscheinlichkeiten für jeden Ort an; genauer als Markov-Filter; lineare Komplexität; Weniger Speicher als Kalman-Filter; Nachteil: Stillstand ohne sich ändernde Sensordaten
\end{itemize}
    \item[Messung:] per Reichweite, Marker in Umgebung
\end{description}
%was sollte hier noch hin

\paragraph*{Visual SLAM,}kurz vSLAM, stellt eine Unterform des \gls{slam} dar, bei der ausschließlich Kameras zur Erfassung der Umgebung eingesetzt werden. Algorithmen verwenden zumeist zusätzlich die Daten der \acrshort{imu}, um die Bewegung der Kamera in die Berechnung der Position einzubeziehen.

\subsection{Stereokamera}\label{chap:stereovision}
Verwendet mehrere Kameras aus parallelverschobenen Bildern Tiefeninformationen zu gewinnen. Der Abstand wird aus markanten Punkten in Bildern zu erkannt. Mithilfe kurzer Mathematik kann die Entfernung zum Punkt ermittelt werden.
\subsection{Optical Flow}
In Bewegungsabläufen werden Objekten verfolgt und können somit relativ zur Kamera bestimmt werden. Das Prinzip wird auch von Lebewesen im Gehirn angewandt. Dabei kann schlecht zwischen der Bewegung der Kamera und der Bewegung von Objekten unterschieden werden. Ungenau, da Kameras immer eine Verzerrung besitzen. 

\subsection{Punktwolkenformat}
Die Möglichkeiten Optical Flow und Stereokamera erzeugen jeweils Tiefenkarten. In diesen Bildern sind, zumeist als Graustufen, Pixel je nach Entfernung zur Kamera gekennzeichnet. Zur Umwandlung als Punktwolke muss jedes Pixel abgetastet werden um als Koordinate im 3D-Raum dargestellt werden zu können. Das \acrshort{ros} beinhaltet sowohl Progamme zur Stereoverarbeitung basierend auf OpenGL\footnote{siehe \url{http://wiki.ros.org/stereo_image_proc}}, als auch die Erzeugung von Punktwolken\footnote{siehe \url{http://wiki.ros.org/depth_image_proc}}.

    \chapter{Einführende Flugtests: Misson-Mode}
Einführend wird die Verwendung von Bodenstation und Drohne im Missionsmodus beschrieben. Zum Einsatz kommt die Software \textit{QGroundcontrol} auf dem PC. 

\begin{multicols}{2}
    Der Missionsplaner kann jederzeit oben Links in \textit{QGroundcontrol}, wie in Bild \ref{fig:qgc_mission_plan} dargestellt, aufgerufen werden. Es öffnet sich ein erweitertes Menü. Nach dem Anlegen des jeweiligen Planes muss dieser zur Drohne hochgeladen werden (nur während eine Verbindung hergestellt ist). Anschließend kann der Missions-Planer verlassen und die jeweilige Mission über das wechseln in den Mission-Mode gestartet werden.
    \vfill\null
    \columnbreak
    \begin{figure}[H]
        \centering
        \includegraphics[width=0.4\textwidth]{images/mission_plan_open.png}
        \caption[QGroundControl Missionplaner-Menu]{QGroundControl Missionplaner-Menu wird durch Klick auf Rot umrahmten Button geöffnet}
        \label{fig:qgc_mission_plan}
    \end{figure}
\end{multicols}

\newpage
\section{Flug auf gerade Linie}
Für den einfachsten Fall lassen sich im Missionsplaner eine Liste von Wegpunkten anlegen, siehe dazu Bild \ref{fig:qgc_mission_plan_wp}. Nachdem am linken Rand \enquote{Waypoint} ausgewählt wurde lassen sich diese beliebig auf der Karte platzieren. Der erste Punkt beschreibt den Start, der letzte kann als Landepunkt dienen. Andernfalls kann die Drohne zum Startpunkt zurückkehren oder in letzter Position stehen bleiben.\\
Jeder Wegpunkt beschreibt eine Position im Raum. Die Parameter Höhe, Bewegungsgeschwindigkeit und eine relative Drehung zur Vorwärtsrichtung (Yaw) können für jeden Punkt separat am rechten Rand festgelegt werden.\\
Im unteren Bildbereich ist das Höhenprofil über den Weg eingezeichnet: in Orange gefärbt die jeweilige Flughöhe der Drohne, in Grün der Boden laut \gls{gps} Karte.\\
Mit den im Bild gezeigten Einstellungen wird die Drohne starten, auf gerade Linie nach vorn fliegen und anschließend landen. Dieses Verhalten wurde in der Simulation überprüft. Dazu wurde ein Hilfsprogramm geschrieben, welches die derzeitige Geschwindigkeit, Höhe und Orientierung der Drohne mitschreibt. Eine Auswertung als Diagramm ist zu sehen in Bild \ref{fig:qgc_mission_plan_wp_dia}.

\begin{figure}[h]
    \centering
    \includegraphics[width=0.6\textwidth]{images/mission_plan_mission.png}
    \caption[QGroundControl Missionplaner-Wegpunkte]{QGroundControl Missionplaner-Wegpunkte}
    \label{fig:qgc_mission_plan_wp}
\end{figure}

\begin{figure}[h]
    \centering
    \includegraphics[width=0.9\textwidth]{images/mission_plan_mission_dia.png}
    \caption[Auswertung Missionsplaner-Wegpunkte]{Auswertung Missionsplaner-Wegpunkte: Diagramme geplottet von Matlab. In den oberen Diagrammen zu sehen sind die jeweilige x-/y-Geschwindigkeit und -Position während des Fluges. Positive x-Richtung zeigt nach Norden, positive y-Richtung nach Osten. Im unteren Diagramm die Flughöhe. Da der Simulator von einer flachen Welt ausging, die GPS-Karte aber eine Position in Östterreich annahm, wurde der Landepunkt tiefer als der eigentliche Boden berechnet. Die Drohne konnte diesen Fehler beim Landen bei ca. $30s$ ausgleichen und blieb sicher stehen.}
    \label{fig:qgc_mission_plan_wp_dia}
\end{figure}

\section{Rally-Kartierung}

\section{Flug mit GeoFence}

\section{Flug mit ROI}

\section{Zusammenfassung}



    \chapter{Vorbereitung zur Umsetzung von Avoidance}

Um die Avoidance-Software umzusetzen, wird diese vorher in verschiedenen Modi erprobt. Dazu zählt eine Simulation. Aus dieser können die Strukturen zum Einsatz mit realer Hardware übernommen werden. Als Zwischenschritt wird die \gls{hil}-Simulation aus dem vorhergehenden Projektteil überarbeitet.

\section{Avoidance als Simulation in Software}\label{chap:sim_gazebo}
Das Avoidance-Modul arbeitet eng mit der Simulationssoftware \enquote{Gazebo} zusammen. In der Dokumentation \cite{dronecodestiftungObstacleDetectionAvoidance2023} ist die Simulation umfangreich dokumentiert, die Umsetzung des Projektes auf reale Hardware jedoch kurz gehalten.\\
Ein vollständiges Linux Ubuntu 20.04 wird benötigt, um die Software per Simulation verwenden zu können. Die Installation beinhaltet eine komplette \acrshort{ros}-Distribution mit graphischen Tools. Mithilfe mehrerer, sich gegenseitig aufrufenden, \enquote{Launch-Files} wird die Simulation instrumentiert und, aufeinander abgestimmt, gestartet. Ohne weiteres Zutun muss nur das Hauptscript zur jeweiligen Simulation aufgerufen werden. Folgend aufgelistet die vier primären Software Bestandteile:
\begin{description}
    \item[Simulator:] \textit{Gazebo}, erzeugt virtuelle Umgebung in der sich eine simulierte Drohne befindet, kann direkt mit dem Flugcontroller kommunizieren
    \item[Flugcontroller:] PX4-Flightstack, Software zur Steuerung einer Drohne, ohne echte Hardware wird ein Flugcontroller simuliert der eine Konsole und Netzwerkanbindungspunkte (via \acrshort{mav}) bereitstellt
    \item[mavros:] \acrshort{mav}-zu-\acrshort{ros}-Übersetzung, siehe \cite[Kapitel 5.2]{markusreinErweiterungBestehenderDrohnen2023}
    \item[Avoidance:] Je nach gewähltem Programm wird der \textit{Local Planner}, \textit{Global Planner}, oder \textit{Safe Landing Planner} gestartet. In diesem Projekt wird der \textit{Local Planner} erprobt.
\end{description}
Zusätzliche Programme, die beim Ausführen des \textit{Local Planner} gestartet werden, sind:
\begin{description}
    \item[rqt-reconfigure:] Feineinstellungen zum Algorithmus \textit{Local Planner} (bspw. bevorzugte Richtungskorrektur, minimale Größe erkannter Hindernisse)
    \item[rviz:] Visualisierung der Drohne im 3-Dimensionalen Raum, dargestellt wird Position der Drohne, ihres Ursprungs, Zielposition, geplante Flugroute, Punktwolkendarstellung von Hindernis, eventuell Blaupause von Hindernis (Abhängig von gewählter Umgebung)
\end{description}
Für die Untersuchung der gegebenen Situation wird zusätzliche Software verwendet:
\begin{description}
    \item[topic-explorer:] Detaillierte Informationen zu Nachrichten im \acrshort{ros}-Umfeld
    \item[tftree:] Verknüpfungen im Transformationsbaum zur Bestimmung der Position der Drohne und derer Peripherie im Bezug zum Ursprung 
\end{description}

Die Interaktion der Bestandteile ist in Abbildung \ref{fig:system_sim_origin} noch einmal dargestellt. Auf rechter Seite gezeichnet sind die Bestandteile in der \enquote{\acrshort{ros}-Umgebung}, welche in die Endanwendung übernommen werden können. Die vollen Aufgaben des Simulators wurden eingekürzt, sie sind zu lesen in \cref{chap:intro_simulation} und \cite[Kapitel 3.4.1]{markusreinErweiterungBestehenderDrohnen2023}. Zusätzlich stellt der Simulator Kamerabilder bereit, je nach gewähltem Modus direkt Tiefenbilder oder Stereobilder welche weiter verarbeitet werden müssen. 
\begin{figure}[!ht]
    \centering
    \includegraphics[width=\linewidth]{images/simulation_ros.drawio.png}
    \caption[Systemaufbau der Simulation von Avoidance]{Systemaufbau der Simulation von Avoidance: Links Simulator (Gazebo) und Flugcontroller (reine Software), Rechts Bestandteile des \acrshort{ros}}
    \label{fig:system_sim_origin}
\end{figure}

Eine erste Überprüfung des Algorithmus wird die Simulation wie in der Anleitung\footnote{\url{https://github.com/PX4/PX4-Avoidance}\cite{dronecodestiftungObstacleDetectionAvoidance2023}} beschrieben durchgeführt. Einige Paramameter wurden zur besseren Übersichtlichkeit angepasst, sodass:
\begin{itemize}
    \item \textit{Gazebo} mit graphischer Oberfläche startet
    \item Die korrekte Punktwolken-Topic vom \textit{Local Planner} abonniert wird
    \item Die Punktwolke der Kamerabilder noch einmal gefiltert wird
\end{itemize} 

Die Anfangsbedingungen der Simulation sind eine vor Bäumen stehende Drohne. Das Bild aus der Simulation ist abgedruckt im Anhang unter \ref{fig:sim_gazebo}. Von der Drohne werden 2 virtuelle Kamerabilder erzeugt, aus welchen eine Punktwolke gebildet wird. Die Punktwolke ist in Bild \ref{fig:sim_gazebo_stereo} links dargestellt. Der Algorithmus zur Erzeugung der Punktwolke ist anfällig für Rauschen verursacht durch minimale Bewegungen in den Kamerabildern, sodass im Bild Artefakte enthalten sind. Deshalb wurde die Punktwolke noch einmal mit einem VoxelGrid\footnote{\url{http://wiki.ros.org/pcl_ros/Tutorials/VoxelGrid\%20filtering}\cite{openroboticsDocumentationROSWiki}} gefiltert, abgebildet in \ref{fig:sim_gazebo_stereo} rechts. Für die Verwendung der Punktwolke mit Avoidance macht das Filtern keine Auswirkung, der \textit{Local Planner} kommt auch mit den ursrpünglichen Daten der Punktwolke zurecht.

\begin{figure}[!h]
    \centering
    \subfloat[][Erzeugte Punktwolke]{\includegraphics[width=0.4\textwidth]{images/sim_gazebo_points_stereo.png}}\hfill
    \subfloat[][Erzeugte Punktwolke, gefiltert]{\includegraphics[width=0.4\textwidth]{images/sim_gazebo_points_stereo_voxel.png}}\hfill
    \caption[Punktwolkendarstellung in \textit{rviz}]{Punktwolkendarstellung in \textit{rviz} generiert aus simulierten Bildern}
    \label{fig:sim_gazebo_stereo}
\end{figure}

Zuletzt ist der Transformationsbaum in Bild \ref{fig:sim_gazebo_tftree} abgedruckt. Eine Transformation ist notwendig, um die Position von Objekten (bspw. der Drohne im Raum, Kameras an der Drohne) relativ zu ihrem Bezugspunkt/ Ursprung darzustellen. Die Position der Drohne wird gekennzeichnet durch den Knoten \enquote{fcu} und ist relativ zum Knoten \enquote{local\_origin}. Um alle bekannten Punkte im Raum anpeilen zu können (von einem bekannten Punkt aus), sollten alle Knoten miteinander verknüpft sein. Jedoch besteht der Transformationsbaum hier aus 4 einzelnen Strängen. Dies ist der Konfiguration von \textit{mavros} geschuldet, die wiederrum in \textit{Avoidance} integriert ist. Die Themen \enquote{map} und \enquote{odom} werden von \textit{mavros} zur Navigation genutzt. Die Transformation \enquote{base\_link} wurde in einer neueren Version von \textit{mavros} eingeführt und soll den Mittelpunkt eines Roboters darstellen, wird aber im Projekt nicht genutzt\footnote{\url{https://www.ros.org/reps/rep-0105.html\#base-link}}. Sowohl \textit{mavros} als auch \textit{Avoidance} können mit dem bestehenden Transformationsbaum arbeiten. Wird ein neues Objekt (bspw. Darstellung einer Punktwolke) im Koordinatensystem 
erzeugt, muss auch eine entsprechende Transformation angelegt werden. Für die Tiefenbilder kommt hier das Thema \enquote{camera\_link}, relativ zur Position der Drohne, zum Einsatz.
\begin{figure}[!h]
    \centering
    \includegraphics[width=\linewidth]{images/sim_gazebo_tftree.png}
    \caption{Transformationsbaum tftree bei Simulation von \textit{Local Planner}}
    \label{fig:sim_gazebo_tftree}
\end{figure}

\section{Avoidance als Simulation mit Hardware}
Als nächste Stufe der Simulation kommt der Simulator \enquote{AirSim} von Microsoft zum Einsatz. Mit \textit{AirSim} stehen die gleichen Möglichkeiten zur Simulation wie mit \textit{Gazebo} zur Verfügung. Allerdings, kann \textit{AirSim} für das Projekt nur mit einem Windows PC zum verwendet werden. Dadurch ergeben sich Beschränkungen, wodurch \textit{AirSim} keine Verbindung zum simulierten Flugcontroller aufbauen kann. Jedoch kann der reale Flugcontroller wie in \cite[Kapitel 5]{markusreinErweiterungBestehenderDrohnen2023} als \gls{hil}-Simulation betrieben werden. Die Veränderungen im Aufbau für die Simulation sind in Bild \ref{fig:system_sim_airsim} rot markiert, die Funktionen der Bestandteile bleiben die gleichen wie in \cref{chap:sim_gazebo}. Hinzu kommt der Bordcomputer, der eine netzwerkgebundene Kommunikation mit dem Flugcontroller erlaubt und der Knoten /enquote{AirSim-Node}, durch den Bilder aus der Simulation ausgelesen werden können. Dabei entfällt das Zusammenfügen zweier Stereobilder, denn die Simulation liefert direkte Tiefenbilder.

\begin{figure}[!h]
    \centering
    \includegraphics[width=\linewidth]{images/simulation_ros-Page-2.drawio.png}
    \caption[Systemaufbau der Simulation von Avoidance]{Systemaufbau der Simulation von Avoidance: Links Simulator (AirSim) und Flugcontroller (\gls{hil}), Rechts Bestandteile des \acrshort{ros}}
    \label{fig:system_sim_airsim}
\end{figure}

Zur Änderung der Funktionalität werden die \textit{Launch-Files} und \enquote{airsim\_ros\_pkgs} (Bibliothek zur Drohnensteuerung im \textit{AirSim}-Simulator) nacheinander angepasst:
\begin{enumerate}
    \item \textit{local\_planner\_stereo.launch}:
    \begin{itemize}
        \item Umbenannt zu \textit{local\_planner\_stereo\_airsim.launch}
        \item Entgernen der Funktion von Disparitätsbildern, da Anwendung keinen ersichtlichen Zweck erfüllt
        \item Unteraufruf von \textit{avoidance\_sitl\_stereo\_airsim.launch} anstatt \textit{avoidance\_sitl\_stereo.launch}
    \end{itemize}
    \item \textit{avoidance\_sitl\_stereo.launch}:
    \begin{itemize}
        \item Umbenannt zu \textit{avoidance\_sitl\_stereo\_airsim.launch}
        \item Entfernen der Stereo-Bildverarbeitung, stattdessen Berechnung der Punktwolkendarstellung aus Tiefenbild von \textit{AirSim}
        \item Entfernen der Transformation von \enquote{fcu} auf \enquote{camera\_link}, da Punktwolkendarstellung noch nicht bekannt
        \item Unteraufruf von \textit{avoidance\_sitl\_mavros\_airsim.launch} anstatt \textit{avoidance\_sitl\_mavros.launch}
    \end{itemize}
    \item \textit{avoidance\_sitl\_mavros.launch}:
    \begin{itemize}
        \item Umbenannt zu \textit{avoidance\_sitl\_mavros\_airsim.launch}
        \item Entfernen Startvorgang des \enquote{PX4 SITL} (Software Simulation des Flugcontrollers)
        \item Ändern des Parameters \enquote{fcu\_url} auf lokale IP-Adresse des Onboard-Computers
        \item Ändern des Parameters \enquote{gcs\_url} auf lokale IP-Adresse des Simulations-PC
        \item Entfernen Startvorgang von \textit{Gazebo}
        \item Einfügen Startvorgang der \enquote{AirSim-Node} (Bestandteil der \textit{airsim\_ros\_pkgs}) zur Darstellung eines simulierten Gefährts, dabei noch wichtige Anpassungen: das erzeugte Tiefenbild muss im Untertopic wie die zugehörige \enquote{camera\_info} vorzufinden sein; der Ursprung aller Ortstransformationen, gennant \enquote{world\_frame\_id} wird entsprechend an \textit{local\_origin} angeknüpft
    \end{itemize}
    \item Einstellungen \textit{AirSim}:
    \begin{itemize}
        \item Anlegen eines \enquote{vehicle}, hier eine Drohne \textit{PX4} die über USB mit der realen Hardware verbunden ist
        \item Anlegen einer Kamera innerhalb der Drohne, hier vom Typ \enquote{DepthPlanar}, gennant \enquote{mk1}
    \end{itemize}
\end{enumerate}

Das ganze funktioniert so nicht... wer hätts gedacht

\clearpage
    \section{Machbarkeitsstudie Stereokamera mit Raspberry Pi}
Vor der Anschaffung eines Kameramoduls für den Raspberry Pi, soll die Nutzbarkeit bewertet werden. Die Akzeptanzkriterien sind:
\begin{itemize}
    \item flüssige Ausgabe von Punktwolke (Avoidance benötigt $10-20$ \gls{fps})
    \item Erfassung großer Gegenstände, kleine Texturen können je nach Kamera und Lichtverhältnissen nicht erfasst werden
    \item Erfassung von Gegenständen im Nahbereich vor der Kamera, weitläufiger Hintergrund kann vernachlässigt werden
\end{itemize}

\acrshort{ros} stellt eine Beispiel-Videoaufnahme mit einer Auflösung von $640x480$ Pixeln und $15$ \gls{fps} Bildwiederholrate bereit, die von 2 Kameras als linke und rechte Perspektive aufgenommen wurde. Diese wird als Referenz für die Tests verwendet.

\subsection*{Durchführung}
\acrshort{ros} stellt die notwendige Software zur Verfügung. Mit dem Programm \textit{stereo\_image\_proc} können 2 Bilder (Linkes und Rechtes Bild) zu einem Tiefenbild zusammengeführt werden.

Die Software \acrshort{ros} kann als komplettes Packet von Docker\footnote{\url{https://hub.docker.com/_/ros/tags}} bezogen werden. Für den \gls{rpi} steht sie in der Version Noetic sowohl für \textit{arm32} als auch \textit{arm64v8} zur Verfügung, die Version wird entsprechend des Betriebssystems ausgewählt. Allerdings sind die Images zur allgemeinen Verwendung bestimmt. Somit enthält keines derer, kompilierte Software mit Optimierungen, die bspw. die Grafikbefehle des \gls{rpi} ausnutzen und so die Bildverarbeitung beschleunigen.

Optimierung steht im Kontext hier für eigene Kompilierung mit den C++-Compiler Flags \texttt{-march=native -O3}. Durch diese wird eine Anpassung auf den aktuell verwendeten Prozessor aktiviert. Speziell die Bildverarbeitung mittels OpenCV kann durch parallele Verarbeitung mit sog. Streaming-Befehlssätzen, ähnlich einer Grafikkarte, beschleunigt werden.

Um die Rechenkapazität des \gls{rpi} zu bewerten, wurden verschiedene Packete manuell kompiliert. Gleichzeitig wurden verschiedene Videoauflösungen zur Berechnung getestet. Verwendet wurde ein \gls{rpi} Model 4 B mit offiziellem 64-Bit Betriebssystem (Raspberry Pi OS). Der Quellcode zum Aufsetzen des Tests ist hinterlegt im GitHub\footnote{\url{https://github.com/aur20/T3000-autonomous_drone/tree/rpi_ros_bench_stereo}}. Das Vorgehen zum Durchführen der Tests ist beschrieben unter \footnote{\url{http://wiki.ros.org/stereo_image_proc/Tutorials/ChoosingGoodStereoParameters}}. Tabelle \ref{tab:bench_stereo_image_proc} zeigt die Ergebnisse in Bildrate je Einstellung und Auflösung.

\begin{table}[!ht]
    \caption{Benchmark zum Programm \textit{stereo\_image\_proc} auf dem \gls{rpi}}
    \begin{tabularx}{\textwidth}{>{\raggedright\arraybackslash}X|>{\raggedright\arraybackslash}X|>{\raggedright\arraybackslash}X}
    Versuch &   Auflösung normal (640x480)    &   Auflösung halbiert (320x240)\\
    \hline
    Standard Packete    &   6.3 &   14.7\\
    \hline
    Optimiertes OpenCV  &   6.9 &   14.7\\
    \hline
    Optimiertes OpenCV,\newline vision\_opencv und image\_pipeline & 6.7 & 14.6\\
    \label{tab:bench_stereo_image_proc}
    \end{tabularx}
\end{table}

\subsection*{Auswertung}
Selbst die Standardpackete erreichen gute Ergebnisse. Die internen Mechanismen beinhalten anscheinend bereits schon die meisten Optimierungen\footnote{siehe \enquote{Advanced SIMD}\url{https://en.wikipedia.org/wiki/AArch64}}. Zusätzlich wurden von den ursrpünglichen Entwicklern weitere Optimierungen im Zusammenhang mit \acrshort{ros} vorgenommen, die den Transfer von Bildinformationen verbessern.

Die Leistung des \gls{rpi} ist nicht ausreichend um hochauflösende Bilder
zu verarbeiten, aber dies wird auch für das Projekt nicht benötigt. Mit dem verwendeten Format wird das Maximum an Bildrate erreicht, es könnte also ein noch wenig besseres Format verwendet werden.

Der entstandene Aufwand steht in keinem Verhältnis zum Gewinn, in der finalen Anwendung können Standardpackete verwendet werden.

\section{Beschaffung Hardware Stereokamera}
Der \gls{rpi} 4 hätte hardware-technische Unterstützung mehrere Kameras anzuschließen. Allerdings besitzen die Modelle A und B nur immer einen notwendigen Header um ein Kameramodul anzuschließen.

Alternative Boards wurden von folgenden Projekten entwickelt:
\begin{itemize}
    \item Stereopi\footnote{\url{https://stereopi.com/}}
    \item Arducam\footnote{\url{https://www.arducam.com/product-category/stereo-vision-cameras/}}
\end{itemize}

Entscheidender Faktor bei der Beschaffung im Rahmen dieses Projektes ist der Preis, weshalb ein Stereopi gewählt wurde. Zusätzlich wird in jedem Fall ein leistungsstarker \gls{rpi} benötigt. Der aktuelle \textit{Stereopi v2} benötigt ein \textit{\gls{cm4}}, eine verkleinerte Version des \gls{rpi}, die als Aufsatz gedacht ist und selbst keine Anschlüsse besitzt. Beide Gerät sind in \ref{fig:stereopi_intro} abgebildet.

\begin{figure}[!ht]
    \centering
    \subfloat[][Raspberry Pi Compute Module 4 (\url{https://www.raspberrypi.com/products/compute-module-4/?variant=raspberry-pi-cm4001000})]{\includegraphics[width=0.4\textwidth]{images/cm4.png}}\hfill
    \subfloat[][Stereopi v2 (\url{https://wiki.stereopi.com/index.php?title=Main\_Page})]{\includegraphics[width=0.4\textwidth]{images/stereopi.png}}\hfill
    \caption[Compute Module 4 und Stereopi]{Compute Module 4 und Stereopi}
    \label{fig:stereopi_intro}
\end{figure}

Da auf zur Zeit der Entwicklung keine \gls{cm4} auf dem Markt verfügbar sind, wird ein alternatives Board, welches Kompatibel zum \gls{rpi} sein soll, verwendet. Zum Einsatz kommt ein \textit{Radxa Compute Module 3}\footnote{https://wiki.radxa.com/Rock3/CM3}.

Zur Verbindung mit dem \textit{Stereopi v2} wurden 2 Kameras vom Typ \textit{Raspberry Pi Camera Module 2} besorgt. Diese bieten hochauflösende Bilder und Videos.

\subsection{Halterung für Platine und Kameras}
Um den \textit{Stereopi v2} und die Kameras zu verbinden, wurde eine Halterung entworfen und per 3D-Druck gefertigt. Die Modelle werden in Bild \ref{fig:stereopi_halterung} vorgestellt. Diese kann später an der Drohne montiert werden.

\begin{figure}[!ht]
    \centering
    \subfloat[][Hauptkörper für Stereopi-Platine mit RPI Compute Module]{\includegraphics[width=0.4\textwidth]{images/stereopi_aufsatz_body.png}}\hfill
    \subfloat[][Steckaufsatz zur Halterung von 2 Kamera-Modulen]{\includegraphics[width=0.4\textwidth]{images/stereopi_aufsatz_kam.png}}\hfill
    \caption[Halterung Platine Stereopi und 2 Kameras]{Halterung Platine Stereopi und 2 Kameras}
    \label{fig:stereopi_halterung}
\end{figure}

\subsection{Software auf dem Compute Module 4}
Das \gls{cm4} enthält einen verlöteten Flash-Chip mit $32GB$ Festspeicher. Das Betriebssystem kann mithilfe eines weiteren Computers geschrieben werden\footnote{siehe \url{https://wiki.radxa.com/Rock3/installusb-install-radxa-cm3-rpi-cm4-io}}. Weitere Schritte zur Einrichtung wurden als Script im Anhang unter \ref{listing:setup_stereopi.sh} hinterlegt. 

Um die Kameraanbindung des Stereopi zu aktivieren, ist eine speziell angepasste Software auf einem \gls{rpi} notwendig. Diese ist nicht mit dem Compute Module 3 von Radxa kompatibel.

Versuche im Rahmen dieser Arbeit haben gezeigt, dass das gesamte Linux Betriebssystem bereits einen veralteten Kernel verwendet. Es wurde zur Entwicklung minimaler Aufwand betrieben und kein weiterer Support bereitgestellt. Es kann offiziell noch zur letzten Version des 4er-Kernels geupdatet werden. Inoffiziell war es im Rahmen der Arbeit möglich auch den 5er-Kernel zu installieren, jedoch entfielen sämtliche Hardware-Funktionen (z.B. war \gls{wlan} nicht mehr verfügbar).

Für Anpassungen an Hardware-Funktionen muss bei Verwendung des 4er-Kernels die Datei \enquote{config.txt} in der Boot-Partition editiert, und anschließend das Script \enquote{update\_extlinux.sh} ausgeführt werden. Mit dem 5er-Kernel funktioniert dies nicht mehr, aber es gibt auch keine Dokumentation diesbezüglich.

Um die Kameraanbindung zu aktivieren, werden 2 Kanäle \gls{i2c} und 2 Kamerainterfaces benötigt. Diese sind in den Konfigurationsdateien des Compute Module 3 enthalten, müssen aber noch entsprechenden Pins zugewiesen werden.

Das Betriebssystem-Image von Radxa enthält weiterhin keinen (aktivierten) Grafiktreiber um etwaige Kameras auslesen zu können. Entweder muss das offizielle CM3-IO Board verwendet werden (welches auch mehrere Kameraanschlüsse besitzt) oder diese tauchen nach Erkennen einer Kamera automatisch auf.\newline

Im Rahmen dieser Arbeit war es nicht möglich die Kameras zu verwenden, zum Einsatz kommen nur die Ultraschallsensoren.

\section{Einrichtung der Ultraschallsensoren}
In diesem Kapitel wird die Verwendung der Ultraschallsensoren in Verbindung mit \acrshort{ros} beschrieben.

\subsection{Erzeugen einer Punktwolke}
Ein Beispiel zur Erzeugung von Punktwolken mit Python ist gegeben unter \footnote{\url{https://docs.ros.org/en/noetic/api/rospy_tutorials/html/publish__pointcloud2_8py_source.html}}, es erzeugt das Bild \ref{fig:ultra_pc_ex}. (Dieses bewegt sich gleichzeitig noch wellenartig.)

Das \acrshort{ros}-Punktwolkenformat definiert eine Klasse mit den Eigenschaften:
\begin{description}
    \item[Header:]
    \begin{itemize}
        \item[]
        \item[$\cdot$] stamp - aktueller Zeitstempel
        \item[$\cdot$] frame\_id - Kennzeichnung zur Transformation
    \end{itemize}
    \item[Fields:] Liste von Eigenschaften der Punkte, beihaltet x-,y-,z-Position sowie Farbattribute
    \item[Points:] Liste von Punkten, jeder Punkt im Format wie \enquote{Fields}
\end{description}

Der Zeitstempel wird mit jeder publizierten Nachricht auf die aktuelle Zeit angepasst.

Für die Anwendung im Projekt ist muss die \enquote{frame\_id} eine entsprechende Transformation von der lokalen Position der Drohne besitzen. Im einfachsten Fall wird die Drohne selbst als Ursprungspunkt des Bildes angenommen sodass keine Transformation benötigt wird. Somit entspricht die verwendete \enquote{frame\_id} vorerst immer \enquote{fcu}.

Als Fields kommen die 3 Raumkoordinaten und ein Farbfeld (Achtung hier Abweichung vom Beispiel) zum Einsatz. Das Farbfeld dient der Darstellung in \textit{rviz}, spielt aber für den Einsatzzweck keine Rolle.\\

Zur Erprobung wird eine Liste von Punkten angelegt. Diese befinden sich in 4 Ecken um den Koordinatenursprung (\enquote{fcu}) und einmal direkt darüber. Die Abstände sind jeweils $1m$ in x- und y-Richtung sowie $1m$ in z-Richtung für den z-Punkt. Die Topic der Nachricht wird mit $4Hz$ veröffentlicht, was der Update Rate der Sensoren entspricht \cite[Kapitel 4.4]{markusreinErweiterungBestehenderDrohnen2023}. Das Script ist im Anhang \ref{listing:pcl_test.py} abgedruckt, das Ergebnis zu sehen hier in Bild \ref{fig:ultra_pc_test}. Zur Zeichnung im Bild gibt es noch keine Transformation. Das Koordinatensystem \enquote{fcu} wird daher als Ursprung angenommen (die Einstellung ist oben links in \textit{rviz} zu sehen).

\begin{figure}[!h]
    \centering
    \includegraphics[width=0.7\linewidth]{images/ultra_pc_test.png}
    \caption[Erprobung von Punktwolkenformat]{Erprobung von Punktwolkenformat in \textit{rviz}: die Punkte jeweils im Abstand von $1m$ zum Ursprung. Als Referenz wurde noch ein Koordinatensystem am Ursprung eingezeichnet, bei diesem enspricht rot der x-Achse=rot, grün der y-Achse, blau der z-Achse}
    \label{fig:ultra_pc_test}
\end{figure}

Zur weiteren Erprobung wird die Simulation gestartet. Die 5 erzeugten Punkte sollten sich gemeinsam mit der Drohne bewegen. In Bild \ref{fig:ultra_pc_test_sim} wurde die Software-Simulation mit Stereokamera gestartet. Anschließend wurde das Script zur Erzeugung von Punktwolken gestartet und in \textit{rviz} hinzugefügt. Zu sehen ist die \enquote{Drohne} (Ursprung der Drohne) mit Punkten, welche sich mitbewegen. Die Bewegung der Punkte ist durch die geringe Update Rate immer etwas später als die der Drohne. Dies sollte aber kein Problem sein, denn es spiegelt das reale Verhalten (die Punkte bleiben stehen aber die Drohne bewegt sich auf diese zu/weg) wieder.

\begin{figure}[!h]
    \centering
    \includegraphics[width=0.7\linewidth]{images/ultra_pc_test_sim.png}
    \caption[Erprobung von Punktwolkenformat mit Simulator]{Erprobung von Punktwolkenformat mit Simulator: die Drohne befindet sich in der Luft vor den Bäumen, um sie herum sind die erzeugten Punkte eingezeichnet}
    \label{fig:ultra_pc_test_sim}
\end{figure}

\subsection{Software der Sensoren}\label{chap:arduino_sensors}
Der Algorithmus zum Auslesen der Sensoren wurde bereits für \cite[Kapitel 4.4]{markusreinErweiterungBestehenderDrohnen2023} entworfen, aber die Anwendung noch nicht dokumentiert. Der Code des Projektteils ist auf GitHub unter \footnote{\url{https://github.com/aur20/T3000-autonomous_drone/tree/arduino_sensors}} hinterlegt. Er besteht aus 2 Teilen:

Die \textbf{\large Arduino Firmware} liest die Sensoren aus und stellt Daten per \gls{i2c} zur Verfügung.

Das Auslesen der Sensoren ist mit der notwendigen Verzögerung verbunden, um eventuellen Echos von Ultraschall vorzubeugen. In \ref{listing:ultra_arduino_loop} dargestellt ist ein Ausschnitt aus der loop()-Schleife, die fortwährend immer durchlaufen wird. Der gezeigte Code kommt 4-mal vor, jeweils mit geänderten Pins der Sensoren (Zeile 187) und geänderten Indizes (Zeile 188-190). In der ersten gezeigten Zeile wird der jeweilige Sensor ausgelesen. Dazu wird mittels einer Arduino-internen Funktionen bis zum empfangenen Echo gewartet und die Zeit zurückgegeben. Anschließend wird der gefilterte Sensorwert berechnet. Der ungefilterte und gefilterte Wert werden jeweils in ein Array eingetragen. Zuletzt erfolgt das Warten, mit einer groben Annäherung: das Makro $PAUSE\_MEAS$ steht für $60ms$. Davon abgezogen wird die Zeit, die bereits auf das Echo gewartet wurde in Millisekunden. Arduino selbst stellt eine Verzögerungsfunktion in Mikrosekunden bereit, dieses sollte jedoch nicht für Verzögerungen länger als \enquote{a few thousend microseconds}\footnote{\url{https://www.arduino.cc/reference/en/language/functions/time/delaymicroseconds/}} verwendet werden, und funktioniert auch nur bis zu $16ms$ Verzögerung. Um die Zeit von Mikrosekunden in Millisekunden umzurechnen, muss durch $1000$ geteilt werden. Im Programm wird stattdessen $10$ mal nach rechts geschoben, was einer Division durch $1024$ entspricht. Die Verarbeitung von Schiebeoperationen ist wesentlich schneller als eine Division. Letztere kann im benötigten Zahlenbereich von kleiner $30.000$ zwar mit Integer Variablen umgesetzt werden, benötigt dann aber nach \footnote{\url{https://forum.arduino.cc/t/speed-of-math-operations-particularly-division-on-arduino/90726/5}} immer noch bis zu $15ms$, was die Messfrequenz beeinflussen würde. Außerdem hat der Prozessor bereits Zeit mit der Berechnung des Filters verbracht, sodass die Wartezeit lediglich eine Annäherung an $60ms$ ist, für das Ergebnis spielt dies keine Rolle.  

\begin{listing}[!ht]
    \cccode[firstline=187, lastline=191]{snippets/ultrasonic_arduino.ino}
    \caption{Ausschnitt der Arduino Firmware: loop()-Schleife}
    \label{listing:ultra_arduino_loop}
\end{listing}

Neben dem Auslesen der Sensoren stellt sich der Arduino als \gls{i2c}-Slave zur Verfügung. Er reagiert auf einzelne zugesendete Buchstaben und Zahlen und sendet eine entsprechende Antwort. Derzeit implementiert sind die Funktionen wie in \ref{tab:ultra_arduino_impl_i2c} aufgezeigt.

\begin{table}[!ht]
    \centering
    \caption{Verfügbare Kommandos zum Auslesen der Sensordaten auf Arduino}
    \begin{tabularx}{0.7\textwidth}{ l | l }
    Kommando & Anwort \\ \hline
    eine der Zahlen $1$-$4$ & jeweiliger Sensorwert, gefiltert\\
    Buchstabe \enquote{a} & alle Sensorwerte, gefiltert\\
    Buchstabe \enquote{r} & alle Sensorwerte, ungefiltert
    \label{tab:ultra_arduino_impl_i2c}
    \end{tabularx}
\end{table}

Das \textbf{\large Python Script} auf dem \gls{rpi} gibt Messwerte aus oder speichert diese als \acrshort{csv}-Datei ab.

Derzeit stehen die Scripte \textit{ultrasonic\_i2c\_reader.py} und \textit{ultrasonic\_i2c\_csvwriter.py} zur Verfügung. Sie verhalten sich nahezu gleich. Bei ersterem kann durch die Eingabe eines Zeichens ein bestimmter Wert an den Arduino gesendet werden. Die Antwort wird entsprechend auf der Konsole ausgegeben.\\
Beim zweiten Script kommt nur der Buchstabe \enquote{a} zum Einsatz. Es wird die Antwort geparst und mit aktuellem Zeitstempel in die Datei eingetragen. \ref{listing:ultra_rpi_impl_i2c} zeigt einen Ausschnitt aus der Schleife des zweiten Scriptes. In Zeile 25 werden die Daten per \gls{i2c}-Verbindung gelesen. Um die 16 Byte Binärdaten als Fließkommazahlen zu interpretieren wird das Packet \enquote{struct} verwendet. Im Beispiel werden die Daten in Zeile 28 als Fließkommazahlen im Little-Endian Format interpretiert. Die Funktion \textit{struct.iter\_unpack} liefert anschließend ein iterierbares Objekt (\enquote{Each iteration yields a tuple as specified by the format string.}\footnote{\url{https://docs.python.org/3/library/struct.html}}). Um den Inhalt zu extrahieren wird eine Lambda-Funktion innerhalb der Funktion \textit{map} verwendet, dies hat zur Folge dass jedes Tupel einzeln betrachtet werden kann. Es wird die jeweilige Zahl extrahiert und in einer Liste angelegt. Die Liste wird zusammen mit der derzeitigen Zeit jeweils ausgegeben und in die Datei geschrieben.

\begin{listing}[!ht]
    \pycode[firstline=25, lastline=30]{snippets/ultrasonic_i2c_csvwriter.py}
    \caption{Ausschnitt des Python Scriptes zum Auslesen der Sensordaten}
    \label{listing:ultra_rpi_impl_i2c}
\end{listing}

\subsection{Einbinden der Sensoren}
Als nächster Schritt wurden die vorliegenden Programme miteinander vereinigt.
Vom letzten Script werden jeweils alle 4 Sensoren ausgelesen. Aufgrund der Ungenauigkeit der Messwerte, wurde der Wertebereich nochmals eingeschränkt und beträgt hier $3m$. Größere Werte (beinhaltet Werte außerhalb der Reichweite, die mit $4m$ gesendet werden) werden schlichtweg ignoriert. Die Angaben für die Punktwolke werden auf Meter umgerechnet.

Das Projekt liegt separat auf GitHub\footnote{\url{https://github.com/aur20/T3000-autonomous_drone/tree/rpi_ros_ultrasonic}} und kann innerhalb zukünftiger Container direkt eingebunden werden.

    \chapter{Erweiterte Flugtests: Avoidance}
Für die Flugtests mit realer Hardware kommt wieder ein \gls{rpi} Model 3B+ zum Einsatz. 

Das Einbinden der Ultraschallsensoren aus dem letzten Kapitel in das \textit{Avoidance} Umfeld erfolgt durch das Ändern der Topic in der Startdatei \enquote{local\_planner.launch}


    \chapter{Zusammenfassung und Reflexion}
Ein autonomer Flug der Drohne konnte nicht bekundigt werden. Doch steht diesem von theoretischer Seite nichts im Weg. Mit den angeschlossenen Sensoren kann die Drohne um flächige Hindernisse manövrieren.\newline

Im Projekt wurden Ultraschallsensoren erfolgreich ausgelesen und deren Sensordaten in einer \acrshort{ros}-Umgebung verarbeitet. Die Sensoren liefern nur Messwerte, wenn sie senkrecht auf etwas großflächiges wie eine Wand gerichtet werden. Somit sind sie kaum für den Feldeinsatz geeignet, denn die meisten Hindernisse erfüllen derartige Kriterien nicht. Außerdem müssten weitere Sensoren rundherum an der Drohne angebracht werden um etwaige Flugrichtungen abzudecken. Das Auslesen der Sensoren selbst wurde aus Effizienz-Gründen ausgelagert. Um die Aufnahmegenauigkeit der Sensoren zu verbessern, könnte eine Interrupt gesteuerte Auslösung des Messvorgangs zum Einsatz kommen. Ein Timer könnte nach jeweils $60ms$ (notwendige Wartezeit zwischen den Messugnen) auslösen um die Messung und Berechnungen des Filters durchzuführen.\newline

Um die Schwachstellen der Ultraschallsensoren auszugelichen, sollte eine Stereokamera (bestehend aus 2 Kameras die gemeinsam eine 3-dimensionale Wahrnehmung des Raumes ermöglichen) angebracht werden. Diese ist nicht mit einfachen Mitteln verfügbar sondern erfordert ein komplexes System aus entsprechender Hard- und Software. Die Realisierung der Software ist mit den bisherigen Gegebenheiten erledigt. Zur Umsetzung wurde aber falsche Hardware gekauft. Speziell ließen sich keine 2 Kameras an einen Raspberry Pi anschließen, weshalb ein Substitut gekauft wurde, welches aber nicht unterstützt wird. Im Nachhinein wäre es praktikabler gewesen, einfache \acrshort{usb}-Kameras anschließen.\newline

Das Verhalten der Drohne mit dem Navigationsalgorithmus hat gezeigt, dass diese sich innerhalb der Steuerung nur Vorwärts bewegt. Das Programm geht davon aus, dass sich keine Hindernisse in der sichtbaren Flugbahn befinden. Treten Hindernisse auf, wird ein alternativer Pfad um diese herum gesucht. Weitere Modifikationen könnten durch Gierung der Drohne überprüfen, ob sich Hindernisse seitlich von dieser befinden. Das Verhalten könnte mittels sich dynamisch anpassender Region of Interest umgesetzt werden.\newline

Auf physikalische Zusammenhänge beim Flug der Drohne wurde nicht eingegangen, da am bestehenden System Drohne keine Änderungen vorgenommen werden. Es wurde lediglich die Software angepasst, was die Flugfähigkeit nicht beeinflußt.

    \appendix
    \chapter{Anhang}
\begin{figure}[!ht]
    \centering
    \includegraphics[width=\linewidth]{images/sim_gazebo.png}
    \caption[Ansicht Gazebo Simulator]{Erste Ansicht von Gazebo bei der Simulation von Avoidance: Relativ klein im Vordergrund die simulierte Drohne mit Koordinatenachsen (Rot, Grün, Blau), im Hintergrund die Welt mit Bäumen, auf linker Seite ein Konfigurationsmenü von Gazebo}
    \label{fig:sim_gazebo}
\end{figure}

    % ********************************************************************
    % End of contents
    % ********************************************************************

    \backmatter
    \cleardoublepage
    \printbibliography

    \cleardoublepage
    \listoffigures
    \cleardoublepage
    \listoftables
    \cleardoublepage
    \printnoidxglossaries
    % Appendix, if needed:


\end{document}