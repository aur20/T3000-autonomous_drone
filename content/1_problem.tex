\chapter{Einleitung}
Dieses Projekt dient der Verwirklichung Autonomen Fliegens.

\textbf{Ziele der Arbeit:}
\begin{itemize}
    \item Erweiterung der Sensorfähigkeit der Drohne um Tiefenkamera
    \item Verwendung von Tiefenbildern und Ultraschallsensordaten für Hinderniserkennung 
    \item Autonomer Flug der Drohne um große, flächige Hindernisse wie Häuser oder Automobile unter Verwendung der Software \textit{Avoidance}
\end{itemize}

\textbf{Optionale Erweiterungen:}
\begin{itemize}
    \item Verfeinerungen und Anpassungen der Software \textit{Avoidance}, sodass kleinere, strukturreiche Hindernisse (bspw. Bäume) erkannt und umflogen werden
    \item Einführung kamerabasierte Zielerkennung als Alternative zur GPS-gestützten Zielführung
\end{itemize}

\textbf{Geplantes Vorgehen:}
\begin{enumerate}
    \item Vervollständigen der Simulation aus vorhergehendem Projektteil: die simulierte Drohne soll im Flug allein mit Kamerabildern und der Software \textit{Avoidance} Hindernissen ausweichen 
    \item Beschaffung und Installation Tiefenkamera an realer Drohne
    \item Aufspielen der Software \textit{Avoidance} auf reale Drohne
    \item Flugtests mit Tiefenkamera
    \item Einbinden von Ultraschallsensordaten in Softwareverarbeitung von \textit{Avoidance}
    \item Flugtests mit Tiefenkamera und Ultraschallsensordaten
    \item mehr Details: was funktioniert in der Endanwendung, was sind die Schritte in der Simulation, ,was wird simuliert
\end{enumerate}

Im weiteren Verlauf der Arbeit spielen folgende Begriffe eine wichtige Rolle:\\
\begin{minipage}[t]{\linewidth}
\begin{description}
    \item[Bodenstation:] 
    \item[\gls{imu}:] Zusammenfassung der Sensorn Beschleunigungssensor, Kompass und Gyroskop
    \item[\gls{rpi}:] Einplatinencomputer, eingesetzt als Bordcomputer
    \item[\gls{ros}:] Metabetriebssystem zur Vernetzung von Roboterbestandteilen (Aktor-Sensor-Verknüpfung), Überwachung und Steuerung von Prozessen
    \item[Docker:] Containerisierung von Betriebssystemumgebungen. Beispielsituation:
    \begin{itemize}
        \item auf dem \gls{rpi} läuft Raspberry Pi OS (Linux Debian), genannt Host-Betriebssystem
        \item unabhängig davon kann innerhalb eines Containers Linux Ubuntu installiert werden
        \item im Container sind speziell festgelegte Bibliotheken, Umgebungsvariablen, Speichergeräte und Netzwerke zum Programmbetrieb vorhanden, es können Programme gestartet werden, die nicht auf dem Host-Betriebssystem lauffähig sind
        \item im Container gestartete Programme laufen parallel zu Programmen auf dem Host-Betriebssystem, sie sind trotzdem im Prozessmanager (\textit{top}) von Raspberry Pi OS sichtbar
    \end{itemize}  
    \item[PX4:] Offizielle Software der Dronecode Stiftung mit der verschiedene $\mu C$, embedded System, oder PC zum Steuern von Robotern, Fahrzeugen oder Drohnen ausgestattet werden können
    \item[Avoidance:] Projekt im Umfeld von PX4 zur autonomen Steuerung von Drohnen in unbekanntem Umfeld. Aufgeteilt in 3 Unterprojekte:
    \begin{itemize}
        \item Local Planner
        \item Global Planner
        \item Safe Landing Planner  
    \end{itemize}
    Im derzeitigen Projekt wird der Local Planner umgesetzt.
    \item[...]
\end{description}
\end{minipage}