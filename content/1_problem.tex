% !TeX spellcheck = de_DE

\setchapterpreamble[u]{%
\renewcommand{\dictumwidth}{0.8\linewidth}
\dictum[Auterion Ltd.]{\enquote{[...] people will directly experience the power of air mobility and autonomous systems—where drones will become a tangible, everyday reality.}}
}
\chapter{Einleitung}
Autonomes Fliegen wird in Zukunft einen immer größeren Stellenwert einnehmen, zuerst werden Drohnen bei der Auslieferung von Waren eingesetzt werden \cite{auterionltdHomeDroneDelivery2022}. Forschung, Drohnen mit den notwendigen Fähigkeiten auszustatten, wird an vielen Stellen betrieben. Dieses Projekt dient der Verwirklichung Autonomen Fliegens unter Verwendung bestehender Technologien.\newline

Zusätzliche Sensoren können Daten für Berechnungen bereitstellen, um den Flug einer Drohne zu optimieren. Diese müssen ausgewertet und die Ergebnisse dem in die Flugberechnung eingearbeitet werden. Das Auswerten und Einspielen der Daten ist je nach Umgebung zeitkritisch, wenn es den Flug direkt beeinflußt. Im Idealfall sollten derartige Berechnungen direkt auf der Drohne durchgeführt werden.\newline

Im Projekt kommen weit verbreitete Methoden bevorzugt, denn sie sind kostengünstige und werden eher von Benutzern adaptiert. In den Ausführungen wird eine Drohne modifiziert um mit Sensoren die Umwelt wahrzunehmen, und Hindernisse detektieren zu können. Zum Einsatz kommen im ersten Schritt Ultraschallsensoren. Um das Auflösungsvermögen zu verbessern, wird zusätzlich mit Kameras gearbeitet. Es sollen sowohl flächige Hindernisse wie Häuser als auch schmale Objekte wie Straßenlaternen erkannt und umflogen werden.

% \begin{itemize}
%     \item Erfassen von Ultraschall-Daten zum Detektieren von Hindernissen
%     \item Erfassen von Bildern
%     \item Verarbeiten von Bildern zur Detektieren von Hindernissen
%     \item Berechnung alternativer Flugbahn zur Umgehung von Hindernissen
% \end{itemize}
% \textbf{Ziele der Arbeit:}
% \begin{itemize}
%     \item Erweiterung der Sensorfähigkeit der Drohne um Tiefenkamera
%     \item Verwendung von Tiefenbildern und Ultraschallsensordaten für Hinderniserkennung 
%     \item Autonomer Flug der Drohne um große, flächige Hindernisse wie Häuser oder Automobile unter Verwendung der Software \textit{Avoidance}
% \end{itemize}
% \textbf{Optionale Erweiterungen:}
% \begin{itemize}
%     \item Verfeinerungen und Anpassungen der Software \textit{Avoidance}, sodass kleinere, strukturreiche Hindernisse (bspw. Bäume) erkannt und umflogen werden
%     \item Einführung kamerabasierte Zielerkennung als Alternative zur GPS-gestützten Zielführung
% \end{itemize}

\textbf{Geplantes Vorgehen für den zweiten Projektteil:}
\begin{enumerate}
    \item Vervollständigen der Simulation aus vorhergehendem Projektteil: die simulierte Drohne soll im Flug allein mit Kamerabildern und der Software \textit{Avoidance} Hindernissen ausweichen 
    \item Beschaffung und Installation Tiefenkamera an realer Drohne
    \item Aufspielen der Software \textit{Avoidance} auf reale Drohne
    \item Flugtests mit Tiefenkamera
    \item Einbinden von Ultraschallsensordaten in Softwareverarbeitung von \textit{Avoidance}
    \item Flugtests mit Tiefenkamera und Ultraschallsensordaten
    \item Vertiefung von gewonnen Erkenntnissen
    \item Verbesserung des Algorithmus
\end{enumerate}

%Im weiteren Verlauf der Arbeit spielen folgende Begriffe eine wichtige Rolle:\\
% \begin{minipage}[t]{\linewidth}
% \begin{description}
%     \item[Bodenstation:] 
%     \item[\gls{imu}:] Zusammenfassung der Sensorn Beschleunigungssensor, Kompass und Gyroskop
%     \item[\gls{rpi}:] Einplatinencomputer, eingesetzt als Bordcomputer
%     \item[\gls{ros}:] Metabetriebssystem zur Vernetzung von Roboterbestandteilen (Aktor-Sensor-Verknüpfung), Überwachung und Steuerung von Prozessen
%     \item[Docker:] Containerisierung von Betriebssystemumgebungen. Beispielsituation:
%     \begin{itemize}
%         \item auf dem \gls{rpi} läuft Raspberry Pi OS (Linux Debian), genannt Host-Betriebssystem
%         \item unabhängig davon kann innerhalb eines Containers Linux Ubuntu installiert werden
%         \item im Container sind speziell festgelegte Bibliotheken, Umgebungsvariablen, Speichergeräte und Netzwerke zum Programmbetrieb vorhanden, es können Programme gestartet werden, die nicht auf dem Host-Betriebssystem lauffähig sind
%         \item im Container gestartete Programme laufen parallel zu Programmen auf dem Host-Betriebssystem, sie sind trotzdem im Prozessmanager (\textit{top}) von Raspberry Pi OS sichtbar
%     \end{itemize}  
%     \item[Avoidance:] Projekt im Umfeld von PX4 zur autonomen Steuerung von Drohnen in unbekanntem Umfeld. Aufgeteilt in 3 Unterprojekte:
%     \begin{itemize}
%         \item Local Planner
%         \item Global Planner
%         \item Safe Landing Planner  
%     \end{itemize}
%     Im derzeitigen Projekt wird der Local Planner umgesetzt.
%     \item[...]
% \end{description}
% \end{minipage}
