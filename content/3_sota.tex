% !TeX spellcheck = de_DE

\chapter{Verfügbare Technologien}
\label{chap:sota}

Im Kapitel werden verfügbare Technologien für den autonomen Drohnenflug vorgestellt und ausgewertet. Dabei wird auf verfügbare Hard- und Software Lösungen eingegangen.

Was ist derzeit in Drohnen implementiert?

Wie macht es DJI.

Intel Kamers

[TODO: das hier benutzen \url{https://github.com/PX4/PX4-Avoidance}]

%\subsubsection{A*-Algorithmus}
%Der am weitesten verbreitete Algorithmus zur Wegsuche ist der A*-Algorithmus. Er sucht den kürzesten Weg in einem Graph basierend auf Kantengewichten zwischen Knoten.
%Bei der Anwendung in Computerspielen wird die Karte dazu als Raster aus Knoten dargestellt. Je nach Weglänge ergibt sich ein Kantengewicht für die Verbindung zweier Knoten.
%Die Schwierigkeit bei der Anwendung von A* in realen Systemen ist es, kontinuierliche Umgebungsinformationen auf einem Raster darzustellen. Je höher die Anzahl an Punkten im Raster, umso exponentiell höher ist der benötigte Rechenaufwand. Bei wenigen Punkten sinkt jedoch die Auflösung, was zu groben Fehlern der Flugplanung führen kann.
%
%\subsubsection{D*-Algortihmus}
%\subsubsection{Obstacle Tracing}
%\subsubsection{Dynamic Pathfinding Algorithm}
%\subsubsection{Rapidly-exploring random trees (RRT)}
%
%
