\thispagestyle{empty}
\pagenumbering{gobble}
{
\centering
\vspace{12cm}
{\large Thema}
\vspace{2cm}

Erweiterung bestehender Drohnen um eine Autonomflugfähigkeit

{\Large \textbf{Studienarbeit T3100}}

\vspace{4cm}

des Studienganges Eletrotechnik

an der Dualen Hochschule Baden-Württemberg, Stuttgart

von

Markus Rein

\vspace{2cm}

Abgabedatum: 22.01.2023

\vspace{2cm}

\begin{tabular}{l l}
    Bearbeitungszeitraum & 21.10.22 -- 22.01.2023\\
    Matrikelnummer, Kurs & 6983030, TEL20GR5\\
    Dualer Partner & Infineon Technologies, Neubiberg\\
    Gutachter*in der Dualen Hochschule & Prof. Dr.-Ing. Johannes Moosheimer
\end{tabular}
}
%Gutachter*in der Dualen Hochschule Titel Vorname Nachname
%\emph{(Betreuer*in bzw. Gutachter*in ggfs. bei Projekt- und Studienarbeiten streichen)}
\cleardoublepage
\bigskip
\begin{tabularx}{\textwidth}{|>{\centering\arraybackslash}X >{\centering\arraybackslash}X >{\centering\arraybackslash}X|}
\hline
& Erklärung & \\
& & \\
\multicolumn{3}{|>{\hsize=\dimexpr3\hsize+2\tabcolsep+\arrayrulewidth\relax}X|}{
Ich versichere hiermit, dass ich meine Projektarbeit mit dem Thema: \enquote{Analyse der Anwendungsmöglichkeiten bei der Vermittlung von Datenpaketen zwischen Mikrocontrollern und Terminalcomputern} selbstständig verfasst und keine anderen als die angegebenen Quellen und Hilfsmittel benutzt habe.\newline
Ich versichere zudem, dass die eingereichte elektronische Fassung mit der gedruckten Fassung übereinstimmt.} \\
& & \\
& & \\
.............................. & .............................. & .............................. \\
{\small Ort} & {\small Datum} & {\small Unterschrift}\\
\hline
\end{tabularx}
\cleardoublepage
\section*{Abstrakt}
In diesem Projekt werden die softwareseitigen Möglichkeiten zur Datenübertragung im Kontext von Eingebetteten Systemen diskutiert. Benötigt wird eine generische Schnittstelle zur Steuerung verschiedener integrierter Geräte. Diese müssen mit Anweisungen versorgt werden und Daten zurück liefern.

Es soll ermöglicht werden, ausgehend von verschiedenen Programmierumgebungen Lösungen zu entwickeln und den Mehrfachaufwand der Implementation zu reduzieren.

Im Rahmen des Projektes werden mehrere Serialisierungs-Bibliotheken vorgestellt und getestet. Danach wird eine Architektur als Grundlage entwickelt mit deren Hilfe eine beispielhafte Implementation einer Kommunikationsanwendung durchgeführt wird.

Fokus liegt auf der ausführlichen Betrachtung der Kodierung von Daten mittels der Bibliotheken, da bei dieser Effizienz eine entscheidende Rolle für Eingebettete Systeme spielt.

\section*{Abstract}
Discussed in this project are possibilities with software for data transfer with embedded systems. Required is a generic interface for control of various integrated devices. These have to be impinged with instructions and they have to deliver data.

The project shall enable development from different framework but reduce additional efforts of development.

In the scope of the project multiple serialization-libraries are introduced and tested. Next an architecture is designed which is then used for an exemplary implementation of a communication-application.

Highlight of the project is the detailed reflection of data encoding with serialization with regard to efficiency in embedded systems.
