\chapter{Zusammenfassung und Reflexion}
Ein autonomer Flug der Drohne konnte nicht bekundigt werden. Doch steht diesem von theoretischer Seite nichts im Weg. Mit den angeschlossenen Sensoren kann die Drohne um flächige Hindernisse manövrieren.\newline

Im Projekt wurden Ultraschallsensoren erfolgreich ausgelesen und deren Sensordaten in einer \acrshort{ros}-Umgebung verarbeitet. Die Sensoren liefern nur Messwerte, wenn sie senkrecht auf etwas großflächiges wie eine Wand gerichtet werden. Somit sind sie kaum für den Feldeinsatz geeignet, denn die meisten Hindernisse erfüllen derartige Kriterien nicht. Außerdem müssten weitere Sensoren rundherum an der Drohne angebracht werden um etwaige Flugrichtungen abzudecken. Das Auslesen der Sensoren selbst wurde aus Effizienz-Gründen ausgelagert. Um die Aufnahmegenauigkeit der Sensoren zu verbessern, könnte eine Interrupt gesteuerte Auslösung des Messvorgangs zum Einsatz kommen. Ein Timer könnte nach jeweils $60ms$ (notwendige Wartezeit zwischen den Messugnen) auslösen um die Messung und Berechnungen des Filters durchzuführen.\newline

Um die Schwachstellen der Ultraschallsensoren auszugelichen, sollte eine Stereokamera (bestehend aus 2 Kameras die gemeinsam eine 3-dimensionale Wahrnehmung des Raumes ermöglichen) angebracht werden. Diese ist nicht mit einfachen Mitteln verfügbar sondern erfordert ein komplexes System aus entsprechender Hard- und Software. Die Realisierung der Software ist mit den bisherigen Gegebenheiten erledigt. Zur Umsetzung wurde aber falsche Hardware gekauft. Speziell ließen sich keine 2 Kameras an einen Raspberry Pi anschließen, weshalb ein Substitut gekauft wurde, welches aber nicht unterstützt wird. Im Nachhinein wäre es praktikabler gewesen, einfache \acrshort{usb}-Kameras anschließen.\newline

Das Verhalten der Drohne mit dem Navigationsalgorithmus hat gezeigt, dass diese sich innerhalb der Steuerung nur Vorwärts bewegt. Das Programm geht davon aus, dass sich keine Hindernisse in der sichtbaren Flugbahn befinden. Treten Hindernisse auf, wird ein alternativer Pfad um diese herum gesucht. Weitere Modifikationen könnten durch Gierung der Drohne überprüfen, ob sich Hindernisse seitlich von dieser befinden. Das Verhalten könnte mittels sich dynamisch anpassender Region of Interest umgesetzt werden.\newline

Auf physikalische Zusammenhänge beim Flug der Drohne wurde nicht eingegangen, da am bestehenden System Drohne keine Änderungen vorgenommen werden. Es wurde lediglich die Software angepasst, was die Flugfähigkeit nicht beeinflußt.
